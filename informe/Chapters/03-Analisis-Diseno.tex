\chapter{Análisis y Diseño del Sistema}
\label{cp:analisis-diseno}

Este capítulo presenta el análisis detallado de los requisitos del sitio web corporativo TechSolutions Pro y su diseño basado en la metodología Jesse James Garrett. Se establecen los requisitos funcionales y no funcionales que guían el desarrollo, seguido de la aplicación sistemática de las cinco capas de la metodología para estructurar la solución de manera coherente y centrada en el usuario.

\section{Análisis de Requisitos}

El análisis de requisitos constituye la base fundamental para el desarrollo exitoso del sitio web corporativo. A través de este proceso se identifican y documentan las necesidades específicas que debe satisfacer la aplicación, tanto desde la perspectiva funcional como no funcional.

\subsection{Requisitos Funcionales}

Los requisitos funcionales describen las funcionalidades específicas que debe proporcionar el sitio web TechSolutions Pro para cumplir con sus objetivos corporativos y satisfacer las necesidades de los usuarios.

\subsubsection{RF001 - Presentación de Información Corporativa}
\textbf{Descripción:} El sistema debe mostrar información completa y actualizada sobre TechSolutions Pro, incluyendo misión, visión, valores, historia y servicios ofrecidos.

\textbf{Criterios de aceptación:}
\begin{itemize}
    \item Mostrar misión y visión corporativa de forma prominente
    \item Presentar historia de la empresa de manera cronológica
    \item Listar servicios con descripciones detalladas
    \item Incluir valores corporativos con explicaciones contextuales
\end{itemize}

\subsubsection{RF002 - Gestión de Información del Equipo}
\textbf{Descripción:} El sistema debe presentar información detallada del equipo de desarrollo, incluyendo perfiles individuales con especialidades y experiencia.

\textbf{Criterios de aceptación:}
\begin{itemize}
    \item Mostrar lista completa del equipo con fotos y roles
    \item Proporcionar páginas de detalle para cada miembro
    \item Incluir información de especialidades técnicas
    \item Presentar experiencia y proyectos relevantes
\end{itemize}

\subsubsection{RF003 - Sistema de Contacto}
\textbf{Descripción:} El sistema debe proporcionar un formulario de contacto funcional con validación de campos y confirmación de envío.

\textbf{Criterios de aceptación:}
\begin{itemize}
    \item Formulario con campos: nombre, email, asunto, mensaje
    \item Validación en tiempo real de todos los campos
    \item Mensajes de error específicos para cada tipo de validación
    \item Confirmación visual de envío exitoso
\end{itemize}

\subsubsection{RF004 - Navegación Responsive}
\textbf{Descripción:} El sistema debe proporcionar navegación fluida y consistente entre todas las páginas, adaptándose a diferentes dispositivos.

\textbf{Criterios de aceptación:}
\begin{itemize}
    \item Menú de navegación consistente en todas las páginas
    \item Navegación responsive para móviles (menú hamburguesa)
    \item Indicación visual de página activa
    \item Breadcrumbs para orientación del usuario
\end{itemize}

\subsubsection{RF005 - Perfiles Detallados del Equipo}
\textbf{Descripción:} El sistema debe permitir acceso a información detallada de cada miembro del equipo mediante navegación dinámica.

\textbf{Criterios de aceptación:}
\begin{itemize}
    \item Páginas individuales para cada miembro del equipo
    \item URLs dinámicas basadas en identificadores únicos
    \item Información extendida: biografía, proyectos, tecnologías
    \item Enlaces de retorno y navegación entre perfiles
\end{itemize}

\subsubsection{RF006 - Presentación de Servicios y Tecnologías}
\textbf{Descripción:} El sistema debe mostrar los servicios tecnológicos ofrecidos por la empresa y las tecnologías utilizadas.

\textbf{Criterios de aceptación:}
\begin{itemize}
    \item Catálogo de servicios con descripciones detalladas
    \item Showcase de tecnologías con logotipos y descripciones
    \item Casos de uso y ejemplos de implementación
    \item Información de precios o modalidades de contratación
\end{itemize}

\subsubsection{RF007 - Footer Informativo}
\textbf{Descripción:} El sistema debe incluir un footer consistente con información de contacto, enlaces útiles y presencia en redes sociales.

\textbf{Criterios de aceptación:}
\begin{itemize}
    \item Información de contacto completa (dirección, teléfono, email)
    \item Enlaces a redes sociales funcionales
    \item Mapa del sitio con enlaces principales
    \item Información de derechos de autor y políticas
\end{itemize}

\subsection{Requisitos No Funcionales}

Los requisitos no funcionales establecen las características de calidad que debe cumplir el sistema en términos de rendimiento, usabilidad, compatibilidad y mantenibilidad.

\subsubsection{RNF001 - Usabilidad}
\textbf{Descripción:} El sistema debe proporcionar una interfaz intuitiva y fácil de usar para usuarios de diferentes niveles técnicos.

\textbf{Criterios de medición:}
\begin{itemize}
    \item Tiempo promedio para completar tareas principales: < 3 minutos
    \item Tasa de éxito en navegación: > 90\%
    \item Curva de aprendizaje mínima para usuarios nuevos
    \item Cumplimiento con principios de usabilidad de Nielsen
\end{itemize}

\subsubsection{RNF002 - Rendimiento}
\textbf{Descripción:} El sistema debe cargar rápidamente y responder de manera eficiente a las interacciones del usuario.

\textbf{Criterios de medición:}
\begin{itemize}
    \item Tiempo de carga inicial: < 3 segundos
    \item Tiempo de navegación entre páginas: < 1 segundo
    \item Tamaño del bundle optimizado: < 2MB
    \item Score de Lighthouse Performance: > 90
\end{itemize}

\subsubsection{RNF003 - Responsividad}
\textbf{Descripción:} El sistema debe funcionar correctamente en dispositivos de diferentes tamaños y resoluciones.

\textbf{Criterios de medición:}
\begin{itemize}
    \item Soporte para móviles: 320px - 768px
    \item Soporte para tablets: 768px - 1024px
    \item Soporte para desktop: 1024px+
    \item Funcionalidad completa en todos los breakpoints
\end{itemize}

\subsubsection{RNF004 - Accesibilidad}
\textbf{Descripción:} El sistema debe ser accesible para usuarios con diferentes capacidades y necesidades especiales.

\textbf{Criterios de medición:}
\begin{itemize}
    \item Cumplimiento con WCAG 2.1 nivel AA
    \item Soporte para lectores de pantalla
    \item Navegación completa mediante teclado
    \item Contraste de colores adecuado (ratio 4.5:1 mínimo)
\end{itemize}

\subsubsection{RNF005 - Compatibilidad}
\textbf{Descripción:} El sistema debe funcionar correctamente en los navegadores web más utilizados.

\textbf{Criterios de medición:}
\begin{itemize}
    \item Google Chrome (últimas 3 versiones)
    \item Mozilla Firefox (últimas 3 versiones)
    \item Safari (últimas 2 versiones)
    \item Microsoft Edge (últimas 2 versiones)
\end{itemize}

\subsubsection{RNF006 - Mantenibilidad}
\textbf{Descripción:} El código del sistema debe ser fácil de mantener, actualizar y extender por desarrolladores.

\textbf{Criterios de medición:}
\begin{itemize}
    \item Cobertura de documentación: > 80\%
    \item Seguimiento de convenciones de Angular Style Guide
    \item Modularización adecuada de componentes
    \item Reutilización de código: > 70\%
\end{itemize}

\subsubsection{RNF007 - SEO (Optimización para Motores de Búsqueda)}
\textbf{Descripción:} El sistema debe estar optimizado para ser indexado correctamente por motores de búsqueda.

\textbf{Criterios de medición:}
\begin{itemize}
    \item Meta tags apropiados en todas las páginas
    \item Estructura semántica HTML5
    \item URLs amigables y descriptivas
    \item Score de Lighthouse SEO: > 90
\end{itemize}

\section{Aplicación de la Metodología Jesse James Garrett}

La aplicación sistemática de las cinco capas de la metodología Jesse James Garrett proporciona la estructura conceptual y práctica para el desarrollo del sitio web TechSolutions Pro. Cada capa se construye sobre las decisiones de la anterior, asegurando coherencia y alineación con los objetivos establecidos.

\subsection{Capa 1: Strategy (Estrategia)}

La capa de estrategia establece los fundamentos del proyecto, definiendo claramente los objetivos de negocio y las necesidades de los usuarios objetivo.

\subsubsection{Objetivos de Negocio}
Los objetivos principales de TechSolutions Pro a través de su sitio web corporativo son:

\begin{itemize}
    \item \textbf{Posicionamiento de marca}: Establecer TechSolutions Pro como líder en soluciones tecnológicas innovadoras
    \item \textbf{Generación de leads}: Captar potenciales clientes interesados en servicios tecnológicos
    \item \textbf{Showcase de competencias}: Demostrar capacidades técnicas y experiencia del equipo
    \item \textbf{Credibilidad profesional}: Proyectar imagen de empresa confiable y establecida
    \item \textbf{Atracción de talento}: Atraer profesionales calificados para unirse al equipo
\end{itemize}

\subsubsection{Audiencia Objetivo}
El análisis de audiencia identifica tres segmentos principales:

\begin{itemize}
    \item \textbf{Clientes potenciales}: Empresas que buscan soluciones tecnológicas personalizadas
    \item \textbf{Partners comerciales}: Otras empresas interesadas en colaboraciones estratégicas
    \item \textbf{Candidatos}: Profesionales en tecnología que buscan oportunidades laborales
\end{itemize}

\subsubsection{Análisis Competitivo}
El análisis del entorno competitivo revela oportunidades de diferenciación:

\begin{itemize}
    \item Competidores locales con presencia web limitada
    \item Empresas internacionales con sitios complejos pero poco personalizados
    \item Oportunidad de destacar mediante experiencia de usuario superior
    \item Diferenciación a través de transparencia en el equipo y procesos
\end{itemize}

\subsubsection{Propuesta de Valor}
TechSolutions Pro se diferencia mediante:

\begin{itemize}
    \item Equipo multidisciplinario con experiencia demostrable
    \item Enfoque en tecnologías modernas y metodologías probadas
    \item Transparencia en procesos y comunicación directa
    \item Soluciones personalizadas adaptadas a necesidades específicas
\end{itemize}

\subsection{Capa 2: Scope (Alcance)}

La capa de alcance traduce la estrategia en especificaciones concretas sobre funcionalidades y contenido del sitio web.

\subsubsection{Especificaciones Funcionales}
Las funcionalidades principales identificadas incluyen:

\begin{itemize}
    \item \textbf{Navegación principal}: Sistema de menú responsive con acceso a todas las secciones
    \item \textbf{Formulario de contacto}: Captura de leads con validación robusta
    \item \textbf{Perfiles del equipo}: Presentación detallada de cada miembro
    \item \textbf{Showcase de servicios}: Catálogo interactivo de ofertas tecnológicas
\end{itemize}

\subsubsection{Requerimientos de Contenido}
El contenido necesario se organiza en las siguientes categorías:

\begin{itemize}
    \item \textbf{Contenido corporativo}: Misión, visión, valores, historia
    \item \textbf{Información del equipo}: Biografías, especialidades, proyectos
    \item \textbf{Portafolio de servicios}: Descripciones técnicas, casos de uso
    \item \textbf{Recursos visuales}: Fotografías profesionales, logotipos, iconografía
\end{itemize}

\subsubsection{Priorización de Características}
Las funcionalidades se clasifican según su importancia estratégica:

\begin{itemize}
    \item \textbf{Críticas}: Navegación, información corporativa, formulario de contacto
    \item \textbf{Importantes}: Perfiles del equipo, showcase de tecnologías
    \item \textbf{Deseables}: Blog corporativo, testimonios de clientes, chatbot
\end{itemize}

\subsubsection{Constraints y Limitaciones}
Las restricciones del proyecto incluyen:

\begin{itemize}
    \item \textbf{Temporales}: Desarrollo en un semestre académico
    \item \textbf{Tecnológicas}: Uso obligatorio de Angular 19 y metodología Garrett
    \item \textbf{Presupuestarias}: Sin costos de hosting o servicios externos
    \item \textbf{Académicas}: Enfoque en demostración de competencias técnicas
\end{itemize}

\subsection{Capa 3: Structure (Estructura)}

La capa de estructura organiza la información y define las interacciones del usuario con el sistema.

\subsubsection{Arquitectura de Información}
La estructura del sitio se organiza jerárquicamente:

\begin{itemize}
    \item \textbf{Nivel 1}: Home (página principal)
    \item \textbf{Nivel 2}: About, Team, Contact (páginas principales)
    \item \textbf{Nivel 3}: Team Details (páginas de detalle dinámicas)
\end{itemize}

\subsubsection{Mapa del Sitio}
La navegación sigue la siguiente estructura:

\begin{itemize}
    \item \textbf{Home} → Página de inicio con overview general
    \item \textbf{About} → Información corporativa completa
    \item \textbf{Team} → Lista de miembros del equipo
    \item \textbf{Team Detail} → Perfiles individuales (dinámicos)
    \item \textbf{Contact} → Formulario y información de contacto
\end{itemize}

\subsubsection{Flujos de Usuario}
Los principales flujos de navegación identificados son:

\begin{itemize}
    \item \textbf{Flujo de información}: Home → About → Contact
    \item \textbf{Flujo de equipo}: Home → Team → Team Detail → Contact
    \item \textbf{Flujo de contacto directo}: Cualquier página → Contact
\end{itemize}

\subsubsection{Diseño de Interacción}
Las interacciones principales incluyen:

\begin{itemize}
    \item Navegación mediante clicks en menú principal
    \item Hover effects en elementos interactivos
    \item Validación en tiempo real en formularios
    \item Transiciones suaves entre páginas
\end{itemize}

\subsection{Capa 4: Skeleton (Esqueleto)}

La capa de esqueleto define la disposición y priorización de elementos en cada página.

\subsubsection{Wireframes de Páginas Principales}
Cada página cuenta con wireframes específicos que definen:

\begin{itemize}
    \item \textbf{Home}: Hero section, servicios destacados, call-to-action
    \item \textbf{About}: Timeline de historia, misión/visión, valores
    \item \textbf{Team}: Grid de miembros, filtros por especialidad
    \item \textbf{Team Detail}: Perfil completo, habilidades, experiencia
    \item \textbf{Contact}: Formulario, mapa, información de ubicación
\end{itemize}

\subsubsection{Diseño de Navegación}
El sistema de navegación incluye:

\begin{itemize}
    \item Header fijo con logo y menú principal
    \item Menú hamburguesa para dispositivos móviles
    \item Breadcrumbs para orientación contextual
    \item Footer con enlaces secundarios y redes sociales
\end{itemize}

\subsubsection{Layout Responsive}
La estructura responsive se adapta mediante:

\begin{itemize}
    \item Grids flexibles que se reorganizan según el viewport
    \item Flexbox para alineación y distribución de elementos
    \item Breakpoints definidos para móvil, tablet y desktop
    \item Imágenes y media escalables
\end{itemize}

\subsubsection{Elementos de Interfaz}
Los componentes de UI principales incluyen:

\begin{itemize}
    \item Botones con estados hover y active
    \item Cards para presentación de información
    \item Inputs con validación visual
    \item Modales para información adicional
\end{itemize}

\subsection{Capa 5: Surface (Superficie)}

La capa de superficie define los aspectos visuales y sensoriales de la experiencia del usuario.

\subsubsection{Diseño Visual}
El diseño visual se caracteriza por:

\begin{itemize}
    \item Paleta de colores corporativa con gradientes púrpura-azul
    \item Tipografía moderna y legible (Inter font family)
    \item Espaciado consistente basado en sistema de grillas
    \item Iconografía coherente utilizando Font Awesome
\end{itemize}

\subsubsection{Paleta de Colores}
Los colores principales del sitio incluyen:

\begin{itemize}
    \item \textbf{Primarios}: Gradientes de púrpura (\#6366f1) a azul (\#3b82f6)
    \item \textbf{Secundarios}: Grises para texto y fondos (\#64748b, \#f1f5f9)
    \item \textbf{Acentos}: Verde para éxito (\#10b981), rojo para errores (\#ef4444)
\end{itemize}

\subsubsection{Tipografía}
La jerarquía tipográfica establece:

\begin{itemize}
    \item \textbf{Headings}: Inter Bold en tamaños escalonados (h1: 2.5rem, h2: 2rem, etc.)
    \item \textbf{Body text}: Inter Regular 1rem con line-height 1.6
    \item \textbf{UI text}: Inter Medium para botones y elementos interactivos
\end{itemize}

\subsubsection{Iconografía}
El sistema de iconos incluye:

\begin{itemize}
    \item Font Awesome para iconos funcionales
    \item Iconos de tecnologías (Angular, TypeScript, etc.)
    \item Iconos de redes sociales consistentes
    \item Tamaños estandarizados (16px, 24px, 32px)
\end{itemize}

La aplicación sistemática de la metodología Jesse James Garrett asegura que cada decisión de diseño esté fundamentada en objetivos claros y necesidades identificadas, resultando en una experiencia de usuario coherente y efectiva para el sitio web corporativo TechSolutions Pro.
