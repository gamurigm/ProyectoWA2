\chapter{Introducción}
\label{cp:introduccion}

{
\parindent0pt

\textit{Proyecto: TechSolutions Pro - Sitio Web Corporativo}

\textit{Autores: Gabriel Murillo y Pablo Zurita}

\textit{Materia: Aplicaciones Web - V Nivel}

\textit{Universidad de las Fuerzas Armadas ESPE}

\vspace{.935em}

Este proyecto presenta el desarrollo de un sitio web corporativo para TechSolutions Pro, aplicando la metodología Jesse James Garrett de User Experience Design en conjunto con Angular 19 como framework de desarrollo. El trabajo se enmarca en la materia de Aplicaciones Web del quinto nivel de la carrera de Ingeniería en Desarrollo de Software de la Universidad de las Fuerzas Armadas ESPE.
}

\section{Contexto}

El desarrollo web moderno ha evolucionado significativamente en las últimas décadas, pasando de sitios estáticos básicos a aplicaciones web complejas e interactivas. En este contexto, la experiencia del usuario (UX) se ha convertido en un factor determinante para el éxito de cualquier proyecto digital. La metodología Jesse James Garrett, establecida en el año 2000 a través de su obra ``The Elements of User Experience'', proporciona un marco estructurado de cinco capas que permite abordar el diseño centrado en el usuario de manera sistemática.

Angular, por su parte, ha emergido como uno de los frameworks más robustos para el desarrollo de Single Page Applications (SPA), y su versión 19 introduce características innovadoras como los componentes standalone, que simplifican la arquitectura y mejoran la mantenibilidad del código. La convergencia entre metodologías UX probadas y tecnologías modernas de desarrollo representa una oportunidad única para crear experiencias digitales excepcionales.

\section{Problemática}

En el ámbito académico del desarrollo web, frecuentemente se observa una desconexión entre la teoría del diseño de experiencia de usuario y su aplicación práctica en proyectos reales. Los estudiantes tienden a enfocarse exclusivamente en aspectos técnicos del desarrollo, descuidando la importancia de una metodología estructurada que garantice una experiencia de usuario coherente y efectiva.

Esta problemática se manifiesta en varios aspectos:

\begin{itemize}
    \item Falta de estructura metodológica en el proceso de diseño y desarrollo
    \item Desconocimiento de la aplicación práctica de metodologías UX establecidas
    \item Ausencia de integración entre principios de diseño centrado en el usuario y frameworks modernos
    \item Limitada experiencia en la implementación de las mejores prácticas de Angular 19
\end{itemize}

\section{Justificación}

La elección de Angular 19 como framework de desarrollo se fundamenta en su madurez tecnológica, robustez en el desarrollo de aplicaciones empresariales, y las innovaciones introducidas en su versión más reciente. Los componentes standalone, las mejoras en el sistema de señales (signals), y la optimización del bundle size convierten a Angular 19 en una herramienta ideal para proyectos académicos que buscan aplicar las mejores prácticas de la industria.

La metodología Jesse James Garrett, por otro lado, ha demostrado su efectividad a lo largo de más de dos décadas en la industria del diseño web. Su enfoque de cinco capas (Strategy, Scope, Structure, Skeleton, Surface) proporciona un marco sistemático que permite abordar cada aspecto del diseño de experiencia de usuario de manera estructurada y coherente.

La combinación de ambos elementos permite crear un proyecto que no solo demuestra competencias técnicas en desarrollo web moderno, sino que también evidencia la capacidad de aplicar metodologías probadas para crear experiencias de usuario excepcionales.

\section{Objetivo General}

Desarrollar un sitio web corporativo para TechSolutions Pro aplicando la metodología Jesse James Garrett de User Experience Design e implementando las funcionalidades mediante Angular 19 con componentes standalone, demostrando la integración efectiva entre metodologías UX estructuradas y tecnologías web modernas.

\section{Objetivos Específicos}

\begin{enumerate}
    \item Implementar sistemáticamente las cinco capas de la metodología Jesse James Garrett (Strategy, Scope, Structure, Skeleton, Surface) en el desarrollo del sitio web corporativo.
    
    \item Desarrollar componentes standalone de Angular 19 que garanticen modularidad, reutilización y mantenibilidad del código.
    
    \item Crear un diseño responsive con breakpoints adaptativos que proporcione una experiencia de usuario óptima en dispositivos móviles, tablets y desktop.
    
    \item Implementar formularios reactivos con validación robusta utilizando las funcionalidades avanzadas de Angular 19.
    
    \item Establecer una arquitectura de routing eficiente que soporte navegación dinámica y lazy loading para optimizar el rendimiento.
\end{enumerate}

\section{Alcance}

El proyecto comprende el desarrollo de un sitio web corporativo de cinco páginas principales:

\begin{itemize}
    \item \textbf{Home}: Página de inicio con información general de la empresa y call-to-action principales
    \item \textbf{About}: Página institucional con misión, visión, historia y valores corporativos
    \item \textbf{Team}: Presentación del equipo de desarrollo con información de cada miembro
    \item \textbf{Team Detail}: Páginas individuales con perfiles detallados de cada miembro del equipo
    \item \textbf{Contact}: Formulario de contacto con validación y información de ubicación
\end{itemize}

El desarrollo incluye la implementación de componentes reutilizables (header, footer, cards), servicios para gestión de datos, routing dinámico con parámetros, y un diseño visual moderno alineado con las mejores prácticas de UI/UX.

\section{Limitaciones}

El proyecto presenta las siguientes limitaciones establecidas para el ámbito académico:

\begin{itemize}
    \item Los datos del equipo y la empresa son hardcoded, no se conecta a una base de datos real
    \item No se implementa sistema de autenticación ni autorización
    \item El formulario de contacto simula el envío pero no procesa datos reales
    \item No se incluye panel de administración para gestión de contenidos
    \item La funcionalidad se limita a la presentación de información y navegación básica
\end{itemize}

Estas limitaciones permiten enfocar el desarrollo en la aplicación correcta de la metodología Jesse James Garrett y las mejores prácticas de Angular 19, cumpliendo con los objetivos académicos establecidos para la materia de Aplicaciones Web.

